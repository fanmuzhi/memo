%***********************************************************************************************
% Memo Micro SETTINGS
%***********************************************************************************************
\newcommand{\WorkWeek}{1332} % Current Work Week (later we should write a script to decide this)
\newcommand{\BossName}{Arden Chang} % Boss's Name
\newcommand{\BossInitials}{ARDN} % Boss's Initials
\newcommand{\Author}{Qiling Bo} % Author's Name
\newcommand{\AuthorInitials}{QIBO} % Author's Cypress Initials
\newcommand{\MemoNumber}{134} % Memo number
\newcommand{\Subject}{Python Exceptions} % Subject of the memo
\newcommand{\Category}{Test development} % Keywords for the memo
\newcommand{\Distribution}{PZHO, MZFA, WEIQ, ARDN, RQB, BMOCHINA} % Distribution list for the memo

%***********************************************************************************************
% Rules
%***********************************************************************************************

% The following characters play a special role in LaTeX:
%
%        # $ % & ~ _ ^ \ { }
%
%If you simply want the character to be printed just as any other letter, 
%include a \ in front of the character. For example, \$ will produce $ in your output. 

%***********************************************************************************************
% Document SETTINGS
%***********************************************************************************************
\documentclass{article}

\usepackage{amsmath}
\usepackage{amsfonts}
\usepackage{amssymb}
\usepackage{fullpage}
\usepackage{graphicx} % Required to insert images
\usepackage{alltt}
\usepackage{color}
\usepackage{hyperref} % For hyper link
\usepackage{float}
\setlength{\headheight}{11pt}
\usepackage{subfigure}


%***********************************************************************************************
% Make Title
%***********************************************************************************************
\renewcommand{\maketitle}
{
%   \\
%   \vspace{-1.25 in}
    \begin{center}
    \includegraphics[width=2.5in]{./template/CY.jpg}\\
    \vspace{5 mm}
    \large
    {
    \textbf{CYPRESS SEMICONDUCTOR CORPORATION}\\
    \textbf{Internal Correspondence}\\
    }
    \vspace{1 mm}
    \hspace{0.5 in}
    \begin{tabular}{rl}
    \bf Date: & \today\ \hspace{2 in}\textbf{WW:\ }\WorkWeek\\
    \bf To: & \BossName\ (\BossInitials)\\
    \bf Author: & \Author\ (\AuthorInitials)\\
    \bf Author File \#: & \AuthorInitials\#\MemoNumber\\
    \bf Subject: & \Subject\\
    \bf Category: & \Category\\
    \bf Distribution: & \Distribution\\
    \end{tabular}
    \vspace{3 mm}\\
    \hrule
    \end{center}
    
    \thispagestyle{firstpage}
    \pagestyle{normalpage}
}


%***********************************************************************************************
% Font
%***********************************************************************************************
% - Helvetica
%\usepackage[scaled]{helvet}
%\renewcommand*\familydefault{\sfdefault} %% Only if the base font of the document is to be sans serif
%\usepackage[T1]{fontenc}

% - Times
%\usepackage{fontspec}
\usepackage{mathptmx}
\usepackage[T1]{fontenc}
%\setmonofont{DejaVu Sans Mono}

%***********************************************************************************************
% Custom Color
%***********************************************************************************************
\definecolor{gray}{rgb}{0.9,0.9,0.9} %for source code display
\definecolor{dkgreen}{rgb}{0.1333,0.5451,0.1333}

%***********************************************************************************************
% Header and Footer
%***********************************************************************************************
\usepackage{fancyhdr} % Required for custom headers

\fancypagestyle{firstpage}{%
  
  \fancyhf{} % clear all header and footer fields
  \fancyfoot[L]{\small{\bf{\thepage}}}
  \fancyfoot[R]{\Author\ (\AuthorInitials)}
  \renewcommand{\headrulewidth}{0.0pt}%
}

%\pagestyle{fancy}
\fancypagestyle{normalpage}{%
    \fancyhf{} % clear all header and footer fields
    \fancyhead[C]{\small{\Subject}}
    \fancyfoot[L]{\small{\bf{\thepage}}}
    \fancyfoot[R]{\Author\ (\AuthorInitials)}
    
    \renewcommand{\headrulewidth}{0.4pt}
    \renewcommand{\footrulewidth}{0.4pt}
    
    \renewcommand{\headsep}{25pt}
}

%\AtBeginDocument{
%    \thispagestyle{firstpage}
%    }

%***********************************************************************************************
% Source code display Setting
%***********************************************************************************************
\usepackage{listings}
\usepackage{courier}

\lstloadlanguages{Python}
\lstset{
     language=Python,
     keywords={from,import,break,case,catch,continue,else,elseif,end,for,def,global,if,otherwise,persistent,return,switch,try,not,while,True,False,in},
     basicstyle=\footnotesize\ttfamily,
     tabsize=2,                   
     extendedchars=true,         
     breaklines=true,            
     keywordstyle=\color{blue},
     frame=none,                    % tb-for top and bottom, single-for around         
     stringstyle=\color{red}\ttfamily, 
     commentstyle=\color{dkgreen},
     backgroundcolor=\color{gray},
     showspaces=false,           
     showtabs=false,             
     xleftmargin=17pt,
     framexleftmargin=17pt,
     framexrightmargin=5pt,
     framexbottommargin=4pt,     
     showstringspaces=false 
}
% Creates a new command to include a C source, the first parameter is the filename of the script (without .c), the second parameter is the caption
\newcommand{\PythonSource}[2]{
\begin{itemize}
\item[]\lstinputlisting[caption=#2,label=#1]{#1.py}
\end{itemize}
}

%***********************************************************************************************
% Figure 
%***********************************************************************************************
%\MySubFigureTwo{./data/IDAC_hist.pdf}{./data/IDAC_hist_2.pdf}{Dynamic}{HuaXin}{0.2in}{IDAC Histogram}{fig1}{0.5}
\newcommand{\MySubFigureTwo}[8]
{
    \vspace{#5}
    \begin{figure}[H]
    \centering
    \subfigure[#3]
    {
        \includegraphics[scale=#8]{#1}
        \label{fig:a#7}
    }
    \subfigure[#4]
    {
        \includegraphics[scale=#8]{#2}
        \label{fig:b#7}
    }
    \caption[#6]{#6}
    \label{fig:#7}
    \end{figure}
}


%***********************************************************************************************
% Document Begin
%***********************************************************************************************
\begin{document}

\maketitle

%------------------------------------ %
%                                     %
%--------START YOUR MEMO HERE-------- %
%                                     %
%------------------------------------ %
\section{INTRODUCTION}
This is the memo for python exceptions best practise. \\
Exceptions are also called program errors, and error handling is very important for program's robustness and reliability.
I summarized 3 types of program errors for a test program:
\begin{itemize}
    \item
    Syntax Errors. The basic error types for errors like typo issues in programming, or a variable is not defined.
    \item
    Logic Mistakes. Unexpected errors, also called bugs, the program is running without reporting any issue, but the logic is not right. 
    \item
    Expected Errors. User defined errors, for example, user inputs a non-valid character, or communication lost.
\end{itemize}
For syntax errors, it can be found during the coding and unit testing. 
For logic mistakes, it also can be found during the unit testing, and logging the errors. It is not easy to debug if we have more than 3 code layers. 
For expected errors, we need have a complete thoughts of the possible situations, and defined the errors in detail.

\newpage
\section{PRE-DEFINED EXCEPTION}
Python defined the exception in module "exceptions", this module is automatically imported, 
normally we don't need to import it. We can use the dir function to check its submodules and functions.
\begin{lstlisting}
In [1]: import exceptions

In [2]: dir(exceptions)
Out[2]: 
['ArithmeticError',
 'AssertionError',
 'AttributeError',
 'BaseException',
 'BufferError',
 'BytesWarning',
 'DeprecationWarning',
 'EOFError',
 'EnvironmentError',
 'Exception',
 'FloatingPointError',
 'FutureWarning',
 'GeneratorExit',
 'IOError',
 'ImportError',
 'ImportWarning',
 'IndentationError',
 'IndexError',
 'KeyError',
 'KeyboardInterrupt',
 'LookupError',
 'MemoryError',
 'NameError',
 'NotImplementedError',
 'OSError',
 'OverflowError',
 'PendingDeprecationWarning',
 'ReferenceError',
 'RuntimeError',
 'RuntimeWarning',
 'StandardError',
 'StopIteration',
 'SyntaxError',
 'SyntaxWarning',
 'SystemError',
 'SystemExit',
 'TabError',
 'TypeError',
 'UnboundLocalError',
 'UnicodeDecodeError',
 'UnicodeEncodeError',
 'UnicodeError',
 'UnicodeTranslateError',
 'UnicodeWarning',
 'UserWarning',
 'ValueError',
 'Warning',
 'ZeroDivisionError',
 '__doc__',
 '__name__',
 '__package__']
\end{lstlisting}
The dir function listed all the pre-defined exceptions in python, 
and we can use "\_\_bases\_\_" property to check the hierarchy.
\begin{lstlisting}
In [3]: ValueError.__bases__
Out[3]: (StandardError,)
\end{lstlisting}
In this example, the ValueError is derived from StandardError. 
And this is the python pre-defined exception hierarchy:
\begin{lstlisting}
BaseException
 +-- SystemExit
 +-- KeyboardInterrupt
 +-- GeneratorExit
 +-- Exception
      +-- StopIteration
      +-- StandardError
      |    +-- BufferError
      |    +-- ArithmeticError
      |    |    +-- FloatingPointError
      |    |    +-- OverflowError
      |    |    +-- ZeroDivisionError
      |    +-- AssertionError
      |    +-- AttributeError
      |    +-- EnvironmentError
      |    |    +-- IOError
      |    |    +-- OSError
      |    |         +-- WindowsError (Windows)
      |    |         +-- VMSError (VMS)
      |    +-- EOFError
      |    +-- ImportError
      |    +-- LookupError
      |    |    +-- IndexError
      |    |    +-- KeyError
      |    +-- MemoryError
      |    +-- NameError
      |    |    +-- UnboundLocalError
      |    +-- ReferenceError
      |    +-- RuntimeError
      |    |    +-- NotImplementedError
      |    +-- SyntaxError
      |    |    +-- IndentationError
      |    |         +-- TabError
      |    +-- SystemError
      |    +-- TypeError
      |    +-- ValueError
      |         +-- UnicodeError
      |              +-- UnicodeDecodeError
      |              +-- UnicodeEncodeError
      |              +-- UnicodeTranslateError
      +-- Warning
           +-- DeprecationWarning
           +-- PendingDeprecationWarning
           +-- RuntimeWarning
           +-- SyntaxWarning
           +-- UserWarning
           +-- FutureWarning
	   +-- ImportWarning
	   +-- UnicodeWarning
	   +-- BytesWarning
\end{lstlisting}
Based on the hierarchy tree, we can know the error type we want to catch, and the relationship of the exceptions.
Most non-system-exiting exception we used is derived from the "Exception" class. 
The "Exception" class has 2 very useful properties, args and message.
\begin{lstlisting}
In [7]: dir(Exception)
Out[7]: 
['__class__',
 '__delattr__',
 '__dict__',
 '__doc__',
 '__format__',
 '__getattribute__',
 '__getitem__',
 '__getslice__',
 '__hash__',
 '__init__',
 '__new__',
 '__reduce__',
 '__reduce_ex__',
 '__repr__',
 '__setattr__',
 '__setstate__',
 '__sizeof__',
 '__str__',
 '__subclasshook__',
 '__unicode__',
 'args',
 'message']
\end{lstlisting}
We can raise an exception with args and message.
\begin{lstlisting}
In [5]: verr = ValueError(True, 10, "This is an error")
In [6]: verr.args
Out[6]: (True, 10, 'This is an error')

In [9]: verr = ValueError("Oops, Value Error")
In [10]: verr.message
Out[10]: 'Oops, Value Error'
\end{lstlisting}


\newpage
\section{USER-DEFINED EXCEPTION}
This is an example of using an instance of Exception as user-defined exception:
\begin{lstlisting}
In [12]: failure1 = {"code":0x31, "message":"IDD High"}

In [13]: try:
    ...:     raise Exception(failure1["code"], failure1["message"])
    ...: except Exception as ex:
    ...:     print ex.args
    ...: 
(49, 'IDD High')
\end{lstlisting}
We can also have a user-defined class to raise the same type exceptions:
\begin{lstlisting}
In [22]: class DUTError(Exception):
    ...:     def __init__(self, **kvargs):
    ...:         self.code = kvargs.get('code', 0)
    ...:         self.message = kvargs.get('message', 'undefined error message')
    ...:     def __str__(self):
    ...:         return repr(self.message)
    ...: 

In [23]: try:
    ...:     raise DUTError(code=0x31, message="IDD High")
    ...: except Exception as ex:
    ...:     print ex.code, ex.message
    ...: 
49 IDD High
\end{lstlisting}
After we defined the DUTError, we can add instances:
\begin{lstlisting}
IDD_HIGH = DUTError(code=0x31, message="Maximum IDD is out of range")
IDD_LOW  = DUTError(code=0x32, message="Minimum IDD is out of range")
\end{lstlisting}

\newpage
\section{TRACE THE EXCEPTION}
We can use 2 python intergrated module "sys" and "traceback" to track the exception.
\begin{lstlisting}
In [25]: import sys
In [26]: import traceback

In [27]: try:
    ...:     open("not_existing.file", "r")
    ...: except Exception as ex:
    ...:     exc_type, exc_value, exc_traceback = sys.exc_info()
    ...:     print traceback.format_exception(exc_type, exc_value, exc_traceback)
    ...: 
['Traceback (most recent call last):\n', u'  File "<ipython-input-27-d9ad7fa44960>", line 2, in <module>\n    open("not_exsist.file", "r")\n', "IOError: [Errno 2] No such file or directory: 'not_exsist.file'\n"]
\end{lstlisting}
In this example, we raise a "file not existing" error, and we use 2 functions \verb|sys.exc_info()| and \\ \verb|traceback.format_exception()| to track and analyze the exception.
And then we can save the message to a log file.

\newpage
\section{BEST PRACTISE}
\begin{lstlisting}
#!/usr/bin/env python
# encoding: utf-8


import os
import logging
import sys
import traceback


class LOG(object):

    def __init__(self, LOG_FILE="error.log"):
        if os.path.isfile(LOG_FILE):
            os.remove(LOG_FILE)     # remove the log file

        self.logger = logging.getLogger()
        hdlr = logging.FileHandler(LOG_FILE)
        formatter = logging.Formatter('%(asctime)s %(levelname)s %(message)s')
        hdlr.setFormatter(formatter)
        self.logger.addHandler(hdlr)
        self.logger.setLevel(logging.NOTSET)

    def save(self, Message):
        self.logger.error(Message)


class DUTError(Exception):

    def __init__(self, **kvargs):
        self.code = kvargs.get('code', 0)
        self.message = kvargs.get('message', "undefined error message")

    def __str__(self):
        return repr(self.message)


class Errors(object):
    IDD_HIGH = DUTError(code=0x31, message="Maximum IDD is out of range")
    IDD_LOW = DUTError(code=0x32, message="Minimum IDD is out of range")


if __name__ == "__main__":
    log = LOG()
    try:
        # test code, dut fail
        # raise Errors.IDD_LOW
        1 / 0
    except DUTError as e:
        # save test fail
        print "Test Fail:" + str(e.code) + ":" + e.message
    except Exception:
        exc_type, exc_value, exc_traceback = sys.exc_info()
        msg = traceback.format_exception(exc_type, exc_value, exc_traceback)
        log.save(msg)

\end{lstlisting}
%------------------------------------ %
%                                     %
%--------END OF YOUR MEMO HERE------- %
%                                     %
%------------------------------------ %

\end{document}
%***********************************************************************************************
% Document End
%***********************************************************************************************
