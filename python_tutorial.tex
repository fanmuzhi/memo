%***********************************************************************************************
% Memo Micro SETTINGS
%***********************************************************************************************
\newcommand{\WorkWeek}{1307} % Current Work Week (later we should write a script to decide this)
\newcommand{\BossName}{Arden Chang} % Boss's Name
\newcommand{\BossInitials}{ARDN} % Boss's Initials
\newcommand{\Author}{Qiling Bo} % Author's Name
\newcommand{\AuthorInitials}{QIBO} % Author's Cypress Initials
\newcommand{\MemoNumber}{087} % Memo number
\newcommand{\Subject}{} % Subject of the memo
\newcommand{\Category}{} % Keywords for the memo
\newcommand{\Distribution}{PZHO, MZFA, WEIQ, ARDN, RQB, BMOCHINA} % Distribution list for the memo

%***********************************************************************************************
% Rules
%***********************************************************************************************

% The following characters play a special role in LaTeX:
%
%        # $ % & ~ _ ^ \ { }
%
%If you simply want the character to be printed just as any other letter, 
%include a \ in front of the character. For example, \$ will produce $ in your output. 

%***********************************************************************************************
% Document SETTINGS
%***********************************************************************************************
\documentclass{article}

\usepackage{amsmath}
\usepackage{amsfonts}
\usepackage{amssymb}
\usepackage{fullpage}
\usepackage{graphicx} % Required to insert images
\usepackage{alltt}
\usepackage{color}
\usepackage{hyperref} % For hyper link
\usepackage{float}
\setlength{\headheight}{11pt}
\usepackage{subfigure}


%***********************************************************************************************
% Make Title
%***********************************************************************************************
\renewcommand{\maketitle}
{
%   \\
%   \vspace{-1.25 in}
    \begin{center}
    \includegraphics[width=2.5in]{./template/CY.jpg}\\
    \vspace{5 mm}
    \large
    {
    \textbf{CYPRESS SEMICONDUCTOR CORPORATION}\\
    \textbf{Internal Correspondence}\\
    }
    \vspace{1 mm}
    \hspace{0.5 in}
    \begin{tabular}{rl}
    \bf Date: & \today\ \hspace{2 in}\textbf{WW:\ }\WorkWeek\\
    \bf To: & \BossName\ (\BossInitials)\\
    \bf Author: & \Author\ (\AuthorInitials)\\
    \bf Author File \#: & \AuthorInitials\#\MemoNumber\\
    \bf Subject: & \Subject\\
    \bf Category: & \Category\\
    \bf Distribution: & \Distribution\\
    \end{tabular}
    \vspace{3 mm}\\
    \hrule
    \end{center}
    
    \thispagestyle{firstpage}
    \pagestyle{normalpage}
}


%***********************************************************************************************
% Font
%***********************************************************************************************
% - Helvetica
%\usepackage[scaled]{helvet}
%\renewcommand*\familydefault{\sfdefault} %% Only if the base font of the document is to be sans serif
%\usepackage[T1]{fontenc}

% - Times
%\usepackage{fontspec}
\usepackage{mathptmx}
\usepackage[T1]{fontenc}
%\setmonofont{DejaVu Sans Mono}

%***********************************************************************************************
% Custom Color
%***********************************************************************************************
\definecolor{gray}{rgb}{0.9,0.9,0.9} %for source code display
\definecolor{dkgreen}{rgb}{0.1333,0.5451,0.1333}

%***********************************************************************************************
% Header and Footer
%***********************************************************************************************
\usepackage{fancyhdr} % Required for custom headers

\fancypagestyle{firstpage}{%
  
  \fancyhf{} % clear all header and footer fields
  \fancyfoot[L]{\small{\bf{\thepage}}}
  \fancyfoot[R]{\Author\ (\AuthorInitials)}
  \renewcommand{\headrulewidth}{0.0pt}%
}

%\pagestyle{fancy}
\fancypagestyle{normalpage}{%
    \fancyhf{} % clear all header and footer fields
    \fancyhead[C]{\small{\Subject}}
    \fancyfoot[L]{\small{\bf{\thepage}}}
    \fancyfoot[R]{\Author\ (\AuthorInitials)}
    
    \renewcommand{\headrulewidth}{0.4pt}
    \renewcommand{\footrulewidth}{0.4pt}
    
    \renewcommand{\headsep}{25pt}
}

%\AtBeginDocument{
%    \thispagestyle{firstpage}
%    }

%***********************************************************************************************
% Source code display Setting
%***********************************************************************************************
\usepackage{listings}
\usepackage{courier}

\lstloadlanguages{Python}
\lstset{
     language=Python,
     keywords={from,import,break,case,catch,continue,else,elseif,end,for,def,global,if,otherwise,persistent,return,switch,try,not,while,True,False,in},
     basicstyle=\footnotesize\ttfamily,
     tabsize=2,                   
     extendedchars=true,         
     breaklines=true,            
     keywordstyle=\color{blue},
     frame=none,                    % tb-for top and bottom, single-for around         
     stringstyle=\color{red}\ttfamily, 
     commentstyle=\color{dkgreen},
     backgroundcolor=\color{gray},
     showspaces=false,           
     showtabs=false,             
     xleftmargin=17pt,
     framexleftmargin=17pt,
     framexrightmargin=5pt,
     framexbottommargin=4pt,     
     showstringspaces=false 
}
% Creates a new command to include a C source, the first parameter is the filename of the script (without .c), the second parameter is the caption
\newcommand{\PythonSource}[2]{
\begin{itemize}
\item[]\lstinputlisting[caption=#2,label=#1]{#1.py}
\end{itemize}
}

%***********************************************************************************************
% Figure 
%***********************************************************************************************
%\MySubFigureTwo{./data/IDAC_hist.pdf}{./data/IDAC_hist_2.pdf}{Dynamic}{HuaXin}{0.2in}{IDAC Histogram}{fig1}{0.5}
\newcommand{\MySubFigureTwo}[8]
{
    \vspace{#5}
    \begin{figure}[H]
    \centering
    \subfigure[#3]
    {
        \includegraphics[scale=#8]{#1}
        \label{fig:a#7}
    }
    \subfigure[#4]
    {
        \includegraphics[scale=#8]{#2}
        \label{fig:b#7}
    }
    \caption[#6]{#6}
    \label{fig:#7}
    \end{figure}
}


%***********************************************************************************************
% Document Begin
%***********************************************************************************************
\begin{document}

\maketitle

%------------------------------------ %
%                                     %
%--------START YOUR MEMO HERE-------- %
%                                     %
%------------------------------------ %

\section{Introduction}
This memo documented the python programming language tutorial. The benefits for using Python for
testing are: 
\begin{itemize}
\item
Add scripts as interfaces/plug-ins for structured program written in C\#.
\item
To quickly analyze the DUTs and the test data on site with simple script.
\item
Use Django to generate the data visulization of test data.
\end{itemize}
For Python scripts, we can use git-hub or other online service to manage the
script code.  Clone the code in different manufacturing sites.
And keep the scripts synchronize with the changes we make.
Python is a script program which is similar to Perl and PHP. In this tutorial,
we will discuss the
items including: 
\begin{itemize}
\item
Basics. Cover the basic definition of data types, collections, modules of python.
\item
Built in functions. Cover the python built in functions like \verb/str()/, \verb/type()/.
\item
Magic functions in class. Cover the special functions built for class, like \verb/__init__()/
\item
Iteration. Important keywords for python like "\verb/yield/".
\item
Built in modules. Cover the python built in modules like os, re, sys.
\end{itemize}

\section{Basics}

\subsection{Data Types}
\begin{itemize}
\item
\verb/__builtin__/ numeric data types: \verb/int/, \verb/float/, \verb/long/ and \verb/complex/.
The corresponding built-in functions for these data types are:
\verb/int()/, \verb/float()/, \verb/long()/ and
\verb/complex()/.  And 2 extra built-in functions \verb/hex()/ and \verb/bin()/.
Usage example:

\begin{lstlisting}
In [1]: i = 100     # define an int variable

In [2]: print(i)    # get the value
Out[2]: 100

In [3]: type(i)     # use type() to get the variable type, equals to i.__class__
Out[3]: int

In [4]: hex(i)      # use hex() to convert to a hexadecimal string, equals to i.__hex__()
Out[4]: '0x64'

In [5]: bin(i)      # use bin() to convert to a binary string
Out[5]: '0b1100100' # a binary data starts with '0b' or '0B', we can define the int with i = 0b1111.

In [6]: f = 2.0     # define a float variable

In [7]: type(f)    # get the type
Out[7]: float

In [13]: i * 10000000000    
Out[13]: 1000000000000L     # a long data ends with a 'L', we can define the long with i = 100L.

In [14]: c = 1.0+2.0j       # define a complex variable

In [15]: c.real     # get the real part
Out[15]: 1.0

In [16]: c.imag    # get the imag part
Out[16]: 2.0
\end{lstlisting}

And there are also keywords in Python corresponding to the data types:
\verb/int/, \verb/float/, \verb/long/ and \verb/complex/.

\begin{lstlisting}
In [22]: if type(100L) is long:
    ...:     print("100L is long")
    ...: 
100L is long

In [23]: if type(1+2j) is complex:
    ...:     print("1+2j is complex")
    ...: 
1+2j is complex
\end{lstlisting}

\item
 \verb/__builtin__/ string

\item
\verb/__builtin__/ constants: \verb/None/, \verb/True/, \verb/False/. Empty string, tuple, list
or dictionary is also taken as False:

\begin{lstlisting}
IN[1]: if not "": print("\"\" is not true")
OUT[2]: "" is not true 
IN[2]: if not (): print("() is not true")
OUT[2]:() is not true
\end{lstlisting}

and 3 more constants, \verb/__debug__/, \verb/NotImplemented/ and \verb/Ellipsis/.
\end{itemize}

\subsection{Collections}
tuple, list, dictionary(hash table). 
The builtin function range() returns a list
zip() and zip(*[]), (zip and unzip)

\subsection{Function and Class}
function and class best practice.   

\subsection{Exception}
try except else and finally. Raise Error.   

\subsection{Modules}
import, from directory import.  

\subsection{Test}
import doctest
    doctest.testmod()

\section{Built in functions}
str()
type()

\section{Magic functions}
\verb/__init__()/
\verb/__iter__()/

\section{Iteration}
yield

\section{Built in modules}
import re
import os
import sys

%------------------------------------ %
%                                     %
%--------END OF YOUR MEMO HERE------- %
%                                     %
%------------------------------------ %



\end{document}
%***********************************************************************************************
% Document End
%***********************************************************************************************
