%***********************************************************************************************
% Memo Micro SETTINGS
%***********************************************************************************************
\newcommand{\WorkWeek}{1322} % Current Work Week (later we should write a script to decide this)
\newcommand{\BossName}{Arden Chang} % Boss's Name
\newcommand{\BossInitials}{ARDN} % Boss's Initials
\newcommand{\Author}{Qiling Bo} % Author's Name
\newcommand{\AuthorInitials}{QIBO} % Author's Cypress Initials
\newcommand{\MemoNumber}{112} % Memo number
\newcommand{\Subject}{AUTOMATION DEPLOYMENT OF DEVELOPMENT ENVIROMENT} % Subject of the memo
\newcommand{\Category}{trackpad, manufacturing test} % Keywords for the memo
\newcommand{\Distribution}{PZHO, MZFA, WEIQ, ARDN, RQB, BMOCHINA} % Distribution list for the memo

%***********************************************************************************************
% Rules
%***********************************************************************************************

% The following characters play a special role in LaTeX:
%
%        # $ % & ~ _ ^ \ { }
%
%If you simply want the character to be printed just as any other letter, 
%include a \ in front of the character. For example, \$ will produce $ in your output. 

%***********************************************************************************************
% Document SETTINGS
%***********************************************************************************************
\documentclass{article}

\usepackage{amsmath}
\usepackage{amsfonts}
\usepackage{amssymb}
\usepackage{fullpage}
\usepackage{graphicx} % Required to insert images
\usepackage{alltt}
\usepackage{color}
\usepackage{hyperref} % For hyper link
\usepackage{float}
\setlength{\headheight}{11pt}
\usepackage{subfigure}


%***********************************************************************************************
% Make Title
%***********************************************************************************************
\renewcommand{\maketitle}
{
%   \\
%   \vspace{-1.25 in}
    \begin{center}
    \includegraphics[width=2.5in]{CY.jpg}\\
    \vspace{5 mm}
    \large
    {
    \textbf{CYPRESS SEMICONDUCTOR CORPORATION}\\
    \textbf{Internal Correspondence}\\
    }
    \vspace{1 mm}
    \hspace{0.5 in}
    \begin{tabular}{rl}
    \bf Date: & \today\ \hspace{2 in}\textbf{WW:\ }\WorkWeek\\
    \bf To: & \BossName\ (\BossInitials)\\
    \bf Author: & \Author\ (\AuthorInitials)\\
    \bf Author File \#: & \AuthorInitials\#\MemoNumber\\
    \bf Subject: & \Subject\\
    \bf Category: & \Category\\
    \bf Distribution: & \Distribution\\
    \end{tabular}
    \vspace{3 mm}\\
    \hrule
    \end{center}
    
    \thispagestyle{firstpage}
    \pagestyle{normalpage}
}


%***********************************************************************************************
% Font
%***********************************************************************************************
% - Helvetica
%\usepackage[scaled]{helvet}
%\renewcommand*\familydefault{\sfdefault} %% Only if the base font of the document is to be sans serif
%\usepackage[T1]{fontenc}

% - Times
%\usepackage{fontspec}
\usepackage{mathptmx}
\usepackage[T1]{fontenc}
%\setmonofont{DejaVu Sans Mono}

%***********************************************************************************************
% Custom Color
%***********************************************************************************************
\definecolor{gray}{rgb}{0.9,0.9,0.9} %for source code display
\definecolor{dkgreen}{rgb}{0.1333,0.5451,0.1333}

%***********************************************************************************************
% Header and Footer
%***********************************************************************************************
\usepackage{fancyhdr} % Required for custom headers

\fancypagestyle{firstpage}{%
  
  \fancyhf{} % clear all header and footer fields
  \fancyfoot[L]{\small{\bf{\thepage}}}
  \fancyfoot[R]{\Author\ (\AuthorInitials)}
  \renewcommand{\headrulewidth}{0.0pt}%
}

%\pagestyle{fancy}

\fancypagestyle{normalpage}{%
  
    \fancyhf{} % clear all header and footer fields
    \fancyhead[C]{\small{\Subject}}
    \fancyfoot[L]{\small{\bf{\thepage}}}
    \fancyfoot[R]{\Author\ (\AuthorInitials)}
    
    \renewcommand{\headrulewidth}{0.4pt}
    \renewcommand{\footrulewidth}{0.4pt}
    
    \renewcommand{\headsep}{25pt}
}

%\AtBeginDocument{
%    \thispagestyle{firstpage}
%    }

%***********************************************************************************************
% Source code display Setting
%***********************************************************************************************
\usepackage{listings}
\usepackage{courier}

\lstloadlanguages{Python}
\lstset{
     language=Python,
     keywords={from,import,break,case,catch,continue,else,elseif,end,for,def,global,if,otherwise,persistent,return,switch,try,not,while,True,False,in},
     basicstyle=\footnotesize\ttfamily,
     tabsize=2,                   
     extendedchars=true,         
     breaklines=true,            
     keywordstyle=\color{blue},
     frame=none,                    % tb-for top and bottom, single-for around         
     stringstyle=\color{red}\ttfamily, 
     commentstyle=\color{dkgreen},
     backgroundcolor=\color{gray},
     showspaces=false,           
     showtabs=false,             
     xleftmargin=17pt,
     framexleftmargin=17pt,
     framexrightmargin=5pt,
     framexbottommargin=4pt,     
     showstringspaces=false 
}
% Creates a new command to include a C source, the first parameter is the filename of the script (without .c), the second parameter is the caption
\newcommand{\PythonSource}[2]{
\begin{itemize}
\item[]\lstinputlisting[caption=#2,label=#1]{#1.py}
\end{itemize}
}

%***********************************************************************************************
% Figure 
%***********************************************************************************************
%\MySubFigureTwo{./data/IDAC_hist.pdf}{./data/IDAC_hist_2.pdf}{Dynamic}{HuaXin}{0.2in}{IDAC
%Histogram}{fig1}{0.5}
\newcommand{\MySubFigureTwo}[8]
{
    \vspace{#5}
    \begin{figure}[H]
    \centering
    \subfigure[#3]
    {
        \includegraphics[scale=#8]{#1}
        \label{fig:a#7}
    }
    \subfigure[#4]
    {
        \includegraphics[scale=#8]{#2}
        \label{fig:b#7}
    }
    \caption[#6]{#6}
    \label{fig:#7}
    \end{figure}
}


%***********************************************************************************************
% Document Begin
%***********************************************************************************************
\begin{document}

\maketitle

%------------------------------------ %
%                                     %
%--------START YOUR MEMO HERE-------- %
%                                     %
%------------------------------------ %
\section{INTRODUCTION}
This memo documented the script the test group used to automatically deploy the development environment.
We used to develop softwares, setup softwares on our laptops, and setup softwares and release softwares on our test machines or servers.
Sometime we did a change on the test machine, to make it work or to make it wok faster, but after a while we forgot what we did. 
And sometime we want a new test machine works just like the old one, but we are too lazy to remember all the settings of the old one and repeat on the new one.
So we introduced the automated deployment tool.\\
For Windows 7, we can use:
\begin{itemize}
    \item
    Windows Automated Installation Kit.    
    http://www.microsoft.com/en-us/download/details.aspx?id=5753
    \item
    puppet. 
    http://docs.puppetlabs.com/windows/index.html 
    \item
    chef.
    http://wiki.opscode.com/display/chef/Fast+Start+Guide+for+Windows
\end{itemize}
For Linux, we can use:
\begin{itemize}
    \item
    puppet. Most popular deployment tool on Linux. 
    https://puppetlabs.com/
    \item
    chef. 
    http://gettingstartedwithchef.com/ 
    \item
    fabric. Written by python, light and flexible tool.
\end{itemize}
From the compare of each tool, it will be good to choose puppet or chef, but since their steep learning curve,
in this memo, we used fabric to automatically deploy the environment on a CentOS 64 Mini, we can use the script in appendix to setup the test logs database in several minutes.

\section{BASIC KNOWLEDGE}
Before we start run the script, we need install the operation system and make the network available, so we can deploy the environment by remote ssh connection.
\subsection{Setup Console Resolution}
The default resolution of console is 640*480, we can change it by editing the grub.conf.
\begin{lstlisting}
$vi /boot/grub/grub.conf
\end{lstlisting}
Add \verb/vga=791/ to the bottom 2 line, before the initrd. The file looks like:
\begin{lstlisting}
kernel /vmlinuz-2.6.18-53.1.19.el5PAE ro root=/dev/VolGroup00/LogVol00 console=ttyS0,57600 console=tty0 vga=791
    initrd /initrd-2.6.18-53.1.19.el5PAE.img
\end{lstlisting}
And the 791 stand for resolution 1024*768, refer to:
\begin{lstlisting}
791 - 1024x768,  16 bit
792 - 1024x768,  24 bit
794 - 1280x1024, 16 bit
795 - 1280x1024, 24 bit
\end{lstlisting}

\subsection{Runlevel and Automatic Start Service}
CentOS defines 7 run levels. 
\begin{itemize}
    \item
    Runlevel 0 - The halt runlevel. This is the runlevel at which the system shuts down. For obvious reasons it is unlikely you would want this as your default runlevel. 
    \item
    Runlevel 1 - Causes the system to start up in a single user mode under which only the root user can log in. In this mode the system does not start any networking, X windowing or multi-user services. This run level is ideal for system administrators to perform system maintenance or repair activities. 
    \item
    Runlevel 2 - Boots the system into a multi-user mode with text based console login capability. This runlevel does not, however, start the network. 
    \item
    Runlevel 3 - Similar to runlevel 2 except that networking services are started. This is the most common runlevel for server based systems that do not require any kind of graphical desktop environment. 
    \item
    Runlevel 4 - Undefined runlevel. This runlevel can be configured to provide a custom boot state. 
    \item
    Runlevel 5 - Boots the system into a networked, multi-user state with X Window System capability. By default the graphical desktop environment will start at the end of the boot process. This is the most common run level for desktop or workstation use. 
    \item
    Runlevel 6 - Reboots the system. Another runlevel that, for obvious reasons, you are unlikely to want as your default. 
\end{itemize}
We can check and edit the default run level:
\begin{lstlisting}
$vi /etc/inittab

~
id:3:initdefault
~
\end{lstlisting}
We can change the run level by:
\begin{lstlisting}
$init 5
\end{lstlisting}
For automatic start service for each run level, we can use chkconfig command, type:
\begin{lstlisting}
$chkconfig --list
\end{lstlisting}
If the service is in the list, we can use chkconfig to turn it on and off, and we can define the on and off for particular run level, for example:
\begin{lstlisting}
$chkconfig mysqld on

$chkconfig --level 3 httpd on
\end{lstlisting}

\subsection{Setup Network}
Config files for computer in our network.
\begin{itemize}
    \item
    /etc/sysconfig/network-scripts/ifcfg-eth0
    \begin{lstlisting}
    DEVICE=eth0
    BOOTPROTO=static
    HWADDR=00:25:90:0E:12:94
    IPADDR=172.23.6.12
    NETMASK=255.255.255.0
    GATEWAY=172.23.6.3
    NM_CONTROLLED=yes
    ONBOOT=yes
    TYPE=Ethernet
    UUID=7617d273-da4d-4287-87df-f74467286b66
    \end{lstlisting}
    \item
    /etc/sysconfig/network
    \begin{lstlisting}
    NETWORKING=yes
    HOSTNAME=BMOCHINA
    GATEWAY=172.23.6.3
    \end{lstlisting}
    \item
    /etc/resolv.conf
    \begin{lstlisting}
    nameserver 172.23.5.62
    nameserver 172.23.5.63
    \end{lstlisting}
\end{itemize}
After change the config files, we need start or restart the network.
\begin{lstlisting}
$service network restart
\end{lstlisting}
Then we can use the ifconfig to check the network status.
\begin{lstlisting}
$ifconfig

eth0      Link encap:Ethernet  HWaddr 3c:97:0e:5b:55:c5  
          inet addr:172.23.6.12  Bcast:172.23.6.255  Mask:255.255.255.0
          inet6 addr: fe80::3e97:eff:fe5b:55c5/64 Scope:Link
          UP BROADCAST RUNNING MULTICAST  MTU:1500  Metric:1
          RX packets:3223261 errors:0 dropped:0 overruns:0 frame:0
          TX packets:111765 errors:0 dropped:0 overruns:0 carrier:0
          collisions:0 txqueuelen:1000 
          RX bytes:580859244 (580.8 MB)  TX bytes:13407651 (13.4 MB)
          Interrupt:20 Memory:f2500000-f2520000 
\end{lstlisting}

\subsection{Setup Iptables}
Before we start the network, another thing we need to do is enable and disable the ports, by config file, /etc/sysconfig/iptables.\\
Here is an example:
\begin{lstlisting}
$iptables -I INPUT 5 -m state --state NEW -m tcp -p tcp --dport 3306 -j ACCEPT
\end{lstlisting}
This command will allow the tcp protocol connection input from port 3306, the state is NEW.
We can set the other ports like this, for example, 80, 22.\\
After we export the rules, we need save the rules, and restart the iptables.
\begin{lstlisting}
$service iptables save 
$service iptables restart 
\end{lstlisting}

\subsection{Setup Tool:rsync}
rysnc is a tool to copy or sync files between 2 machines, 2 folder in 1 machine, or hard disk and USB disk.\\
\begin{itemize}
    \item
    Copy file or folder from local to remote. For example, copy local file to remote machine with IP address"192.168.0.1", put the file in root's home directory.\\
    \begin{lstlisting}
    $rsync -v -e ssh /path/to/file root@192.168.0.1:~
    \end{lstlisting}
    \item
    Copy file or folder from remote to local. For example, copy remote file test.txt in root's home directory, and paste to local tmp folder. 
    \begin{lstlisting}
    $rsync -v -e ssh root@192.168.0.1:~/test.txt /tmp 
    \end{lstlisting}
    \item
     Synchronize a local directory with a remote directory. Keep the local files the same as remote, the extra files will be deleted. 
    \begin{lstlisting}
    $rsync -r -a -v -e "ssh -l root" --delete /local/path/tmp 192.168.0.1:/tmp
    \end{lstlisting}
    \item
    Synchronize a remote directory with a local directory. Keep the remote files the same as local, the extra files will be deleted. 
    \begin{lstlisting}
    $rsync -r -a -v -e "ssh -l root" --delete 192.168.0.1:/tmp /local/path/tmp
    \end{lstlisting}
\end{itemize}

\subsection{Setup SELinux}
Security-Enhanced Linux (SELinux) is a Linux feature that provides the mechanism for supporting access control security policies.
SELinux will cause a lot of unexpected authorization failure, but we recommend to turn it on for the safety consideration.\\ 
We can get the status of SELinux by command:
\begin{lstlisting}
$getenforce

Enforcing
\end{lstlisting}
"Enforcing" is the default mode, "Permissive" is used for troubleshooting, and "Disabled" is SElinux turned off.
We can set the status of SELinux to "Enforcing"(1) and "Permissive"(0) by command:
\begin{lstlisting}
$setenforce 0
$getenforce

Permissive
\end{lstlisting}
Get SELinux booleans value by:
\begin{lstlisting}
$getsebool -a
\end{lstlisting}
For the test logs database, we enable the httpd to connect to the database and memcache.
\begin{lstlisting}
$setsebool -P httpd_can_network_connect 1
$setsebool -P httpd_can_network_connect_db 1
$setsebool -P httpd_can_network_memcache 1
$setsebool -P httpd_can_network_relay 1
\end{lstlisting}
If we are still have problem with SELinux with other access authorization, we can check the log file at:
\begin{lstlisting}
$vi /var/log/audit/audit.log
\end{lstlisting}

\section{APPENDIX}
\PythonSource{./fabfile}{Fabric demo script for CentOS 64 Mini}
%------------------------------------ %
%                                     %
%--------END OF YOUR MEMO HERE------- %
%                                     %
%------------------------------------ %

\end{document}
%***********************************************************************************************
% Document End
%***********************************************************************************************
