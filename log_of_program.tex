%***********************************************************************************************
% Memo Micro SETTINGS
%***********************************************************************************************
\newcommand{\WorkWeek}{1324} % Current Work Week (later we should write a script to decide this)
\newcommand{\BossName}{Arden Chang} % Boss's Name
\newcommand{\BossInitials}{ARDN} % Boss's Initials
\newcommand{\Author}{Qiling Bo} % Author's Name
\newcommand{\AuthorInitials}{QIBO} % Author's Cypress Initials
\newcommand{\MemoNumber}{115} % Memo number
\newcommand{\Subject}{PROGRAM LOGS} % Subject of the memo
\newcommand{\Category}{Test Development} % Keywords for the memo
\newcommand{\Distribution}{PZHO, MZFA, WEIQ, ARDN, RQB, BMOCHINA} % Distribution list for the memo

%***********************************************************************************************
% Rules
%***********************************************************************************************

% The following characters play a special role in LaTeX:
%
%        # $ % & ~ _ ^ \ { }
%
%If you simply want the character to be printed just as any other letter, 
%include a \ in front of the character. For example, \$ will produce $ in your output. 

%***********************************************************************************************
% Document SETTINGS
%***********************************************************************************************
\documentclass{article}

\usepackage{amsmath}
\usepackage{amsfonts}
\usepackage{amssymb}
\usepackage{fullpage}
\usepackage{graphicx} % Required to insert images
\usepackage{alltt}
\usepackage{color}
\usepackage{hyperref} % For hyper link
\usepackage{float}
\setlength{\headheight}{11pt}
\usepackage{subfigure}


%***********************************************************************************************
% Make Title
%***********************************************************************************************
\renewcommand{\maketitle}
{
%   \\
%   \vspace{-1.25 in}
    \begin{center}
    \includegraphics[width=2.5in]{CY.jpg}\\
    \vspace{5 mm}
    \large
    {
    \textbf{CYPRESS SEMICONDUCTOR CORPORATION}\\
    \textbf{Internal Correspondence}\\
    }
    \vspace{1 mm}
    \hspace{0.5 in}
    \begin{tabular}{rl}
    \bf Date: & \today\ \hspace{2 in}\textbf{WW:\ }\WorkWeek\\
    \bf To: & \BossName\ (\BossInitials)\\
    \bf Author: & \Author\ (\AuthorInitials)\\
    \bf Author File \#: & \AuthorInitials\#\MemoNumber\\
    \bf Subject: & \Subject\\
    \bf Category: & \Category\\
    \bf Distribution: & \Distribution\\
    \end{tabular}
    \vspace{3 mm}\\
    \hrule
    \end{center}
    
    \thispagestyle{firstpage}
    \pagestyle{normalpage}
}


%***********************************************************************************************
% Font
%***********************************************************************************************
% - Helvetica
%\usepackage[scaled]{helvet}
%\renewcommand*\familydefault{\sfdefault} %% Only if the base font of the document is to be sans serif
%\usepackage[T1]{fontenc}

% - Times
%\usepackage{fontspec}
\usepackage{mathptmx}
\usepackage[T1]{fontenc}
%\setmonofont{DejaVu Sans Mono}

%***********************************************************************************************
% Custom Color
%***********************************************************************************************
\definecolor{gray}{rgb}{0.9,0.9,0.9} %for source code display
\definecolor{dkgreen}{rgb}{0.1333,0.5451,0.1333}

%***********************************************************************************************
% Header and Footer
%***********************************************************************************************
\usepackage{fancyhdr} % Required for custom headers

\fancypagestyle{firstpage}{%
  
  \fancyhf{} % clear all header and footer fields
  \fancyfoot[L]{\small{\bf{\thepage}}}
  \fancyfoot[R]{\Author\ (\AuthorInitials)}
  \renewcommand{\headrulewidth}{0.0pt}%
}

%\pagestyle{fancy}

\fancypagestyle{normalpage}{%
  
    \fancyhf{} % clear all header and footer fields
    \fancyhead[C]{\small{\Subject}}
    \fancyfoot[L]{\small{\bf{\thepage}}}
    \fancyfoot[R]{\Author\ (\AuthorInitials)}
    
    \renewcommand{\headrulewidth}{0.4pt}
    \renewcommand{\footrulewidth}{0.4pt}
    
    \renewcommand{\headsep}{25pt}
}

%\AtBeginDocument{
%    \thispagestyle{firstpage}
%    }

%***********************************************************************************************
% Source code display Setting
%***********************************************************************************************
\usepackage{listings}
\usepackage{courier}

\lstloadlanguages{Python}
\lstset{
     language=Python,
     keywords={from,import,break,case,catch,continue,else,elseif,end,for,def,global,if,otherwise,persistent,return,switch,try,not,while,True,False,in},
     basicstyle=\footnotesize\ttfamily,
     tabsize=2,                   
     extendedchars=true,         
     breaklines=true,            
     keywordstyle=\color{blue},
     frame=none,                    % tb-for top and bottom, single-for around         
     stringstyle=\color{red}\ttfamily, 
     commentstyle=\color{dkgreen},
     backgroundcolor=\color{gray},
     showspaces=false,           
     showtabs=false,             
     xleftmargin=17pt,
     framexleftmargin=17pt,
     framexrightmargin=5pt,
     framexbottommargin=4pt,     
     showstringspaces=false 
}
% Creates a new command to include a C source, the first parameter is the filename of the script (without .c), the second parameter is the caption
\newcommand{\PythonSource}[2]{
\begin{itemize}
\item[]\lstinputlisting[caption=#2,label=#1]{#1.py}
\end{itemize}
}

%***********************************************************************************************
% Figure 
%***********************************************************************************************
%\MySubFigureTwo{./data/IDAC_hist.pdf}{./data/IDAC_hist_2.pdf}{Dynamic}{HuaXin}{0.2in}{IDAC
%Histogram}{fig1}{0.5}
\newcommand{\MySubFigureTwo}[8]
{
    \vspace{#5}
    \begin{figure}[H]
    \centering
    \subfigure[#3]
    {
        \includegraphics[scale=#8]{#1}
        \label{fig:a#7}
    }
    \subfigure[#4]
    {
        \includegraphics[scale=#8]{#2}
        \label{fig:b#7}
    }
    \caption[#6]{#6}
    \label{fig:#7}
    \end{figure}
}


%***********************************************************************************************
% Document Begin
%***********************************************************************************************
\begin{document}

\maketitle

%------------------------------------ %
%                                     %
%--------START YOUR MEMO HERE-------- %
%                                     %
%------------------------------------ %
\section{INTRODUCTION}
The text logs of program bugs and errors are used very often in test program, sometime we print the information on the screen.
But in production, we prefer to save the debug information in the text logs than printing it.
This memo documented the usage and method to read and write program logs, including the program we installed and the test program we coded.

\section{DETAIL}
\subsection{Text Logs in Test Program}
In this section we introduced the test program logs we coded in C\# and python, for TMT, TPT currently and MTK in the future.
Each line of the text log contains 3 columns, "Executed Time" + "Error Level" + "Error Message".
For example: 
\begin{lstlisting}
130605 13:54:08 [ERROR] Plugin 'InnoDB' registration as a STORAGE ENGINE failed. 
\end{lstlisting}
This is a default error message of MySQL, "Executed Time" is "130606 13:54:08", "Error Level" is "ERROR" and "Error Message" is "Plugin...failed.".
So we can follow the information to debug our program and find a solution of our test program problem.
We listed 2 examples of how to write the program logs in C\# and python, as follow.
\begin{itemize}
\item
C\# code example:
\begin{lstlisting}
    class Log
    {
        public static string content;
        public static string sectionContent;

        public static void info(string text)
        {
            writeLog("INFO ", text);
        }

        public static void warning(string text)
        {
            writeLog("WARN ", text);
        }

        public static void error(string text)
        {
            writeLog("ERROR", text);
        }

        public static void error(string text, bool die)
        {
            writeLog("ERROR", text);
            if (die == true)
                Process.GetCurrentProcess().Kill();
        }

        private static void writeLog(string type, string error)
        {
            DateTime myTimeStamp = DateTime.Now;
            string date = myTimeStamp.ToString("yyyy-MM-dd HH:mm:ss");

            string prefix = "[" + date + "] " + type.ToUpper() + " : ";

            content = content + prefix + error + "\r\n";
            sectionContent = sectionContent + prefix + error + "\r\n";

            StreamWriter myFile = new StreamWriter(Directory.GetCurrentDirectory() + "\\ErrorMessage.log", true);
            myFile.Write(prefix + error + "\r\n");
            myFile.Close();
        }
    }
\end{lstlisting}
\item
Python code example:
\begin{lstlisting}
#!/usr/bin/env python
# encoding: utf-8

import os
import logging

LOG_FILE = 'log.txt'

def init_log():
    if os.path.isfile(LOG_FILE):
        os.remove(LOG_FILE)     # remove the log file

    logger = logging.getLogger()
    hdlr = logging.FileHandler(LOG_FILE)
    formatter = logging.Formatter('%(asctime)s %(levelname)s %(message)s')
    hdlr.setFormatter(formatter)
    logger.addHandler(hdlr)
    logger.setLevel(logging.NOTSET)


def main():
    init_log()
    logging.info("test info")
    logging.error("error info")
    logging.debug("debug info")


if __name__ == "__main__":
    main()
\end{lstlisting}
\end{itemize}


\subsection{Text Logs in LAMP}
In this section we introduced the debug logs we used in our database system. 
We can use these log files to debug our program and used the same formmat for our developed test programs.
\subsubsection{MySQL}
\begin{itemize}
    \item
    Error log. MySQL error log is defined in my.cnf, log-error=YourPath.log.\\
    Default location of error log is:
    \begin{lstlisting}
    /var/log/mysqld.log
    \end{lstlisting}
    \item
    Slow Query log. MySQL slow query log is defined in my.cnf, slow\_query\_log.
    Default location of slow query log is:
    \begin{lstlisting}
    /var/run/mysqld/mysqld-slow.log
    \end{lstlisting}
\end{itemize}

\subsubsection{PHP}
PHP error log is defined in php.ini, error\_log=YourPath.log. \\
Default location of php error log is:
\begin{lstlisting}
error_log=php_errors.log
\end{lstlisting}
Notice that the default setting of error recording is turned off, so we need turn it on and set a valid path to it.

\subsubsection{Apache}
\begin{itemize}
    \item
    Default location of apache error log:
    \begin{lstlisting}
    /var/log/httpd/error.log
    \end{lstlisting}
    \item
    Default location of apache access log:
    \begin{lstlisting}
    /var/log/httpd/access.log
    \end{lstlisting}
\end{itemize}

\subsubsection{SELinux}
The SELinux permission log is recorded in audit.log, with a type=AVC, we can list the log and grep the AVC lines.
We can find the permission error information at:
\begin{lstlisting}
/var/log/audit/audit.log
$less /var/log/audit/audit.log | grep AVC
\end{lstlisting}

%\MySubFigureTwo{pic1path}{pic2path}{pic1comment}{pic2comment}{0.2in}{title}{fig1}{0.5}

%------------------------------------ %
%                                     %
%--------END OF YOUR MEMO HERE------- %
%                                     %
%------------------------------------ %

\end{document}
%***********************************************************************************************
% Document End
%***********************************************************************************************
