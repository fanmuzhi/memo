%***********************************************************************************************
% Memo Micro SETTINGS
%***********************************************************************************************
\newcommand{\WorkWeek}{1339} % Current Work Week (later we should write a script to decide this)
\newcommand{\BossName}{Arden Chang} % Boss's Name
\newcommand{\BossInitials}{ARDN} % Boss's Initials
\newcommand{\Author}{Qiling Bo} % Author's Name
\newcommand{\AuthorInitials}{QIBO} % Author's Cypress Initials
\newcommand{\MemoNumber}{118} % Memo number
\newcommand{\Subject}{PACKAGE AND DEPLOY PYTHON PROJECT TO WINDOWS OS} % Subject of the memo
\newcommand{\Category}{trackpad, manufacturing test} % Keywords for the memo
\newcommand{\Distribution}{PZHO, MZFA, WEIQ, ARDN, RQB, BMOCHINA} % Distribution list for the memo

%***********************************************************************************************
% Rules
%***********************************************************************************************

% The following characters play a special role in LaTeX:
%
%        # $ % & ~ _ ^ \ { }
%
%If you simply want the character to be printed just as any other letter, 
%include a \ in front of the character. For example, \$ will produce $ in your output. 

%***********************************************************************************************
% Document SETTINGS
%***********************************************************************************************
\documentclass{article}

\usepackage{amsmath}
\usepackage{amsfonts}
\usepackage{amssymb}
\usepackage{fullpage}
\usepackage{graphicx} % Required to insert images
\usepackage{alltt}
\usepackage{color}
\usepackage{hyperref} % For hyper link
\usepackage{float}
\setlength{\headheight}{11pt}
\usepackage{subfigure}


%***********************************************************************************************
% Make Title
%***********************************************************************************************
\renewcommand{\maketitle}
{
%   \\
%   \vspace{-1.25 in}
    \begin{center}
    \includegraphics[width=2.5in]{./template/CY.jpg}\\
    \vspace{5 mm}
    \large
    {
    \textbf{CYPRESS SEMICONDUCTOR CORPORATION}\\
    \textbf{Internal Correspondence}\\
    }
    \vspace{1 mm}
    \hspace{0.5 in}
    \begin{tabular}{rl}
    \bf Date: & \today\ \hspace{2 in}\textbf{WW:\ }\WorkWeek\\
    \bf To: & \BossName\ (\BossInitials)\\
    \bf Author: & \Author\ (\AuthorInitials)\\
    \bf Author File \#: & \AuthorInitials\#\MemoNumber\\
    \bf Subject: & \Subject\\
    \bf Category: & \Category\\
    \bf Distribution: & \Distribution\\
    \end{tabular}
    \vspace{3 mm}\\
    \hrule
    \end{center}
    
    \thispagestyle{firstpage}
    \pagestyle{normalpage}
}


%***********************************************************************************************
% Font
%***********************************************************************************************
% - Helvetica
%\usepackage[scaled]{helvet}
%\renewcommand*\familydefault{\sfdefault} %% Only if the base font of the document is to be sans serif
%\usepackage[T1]{fontenc}

% - Times
%\usepackage{fontspec}
\usepackage{mathptmx}
\usepackage[T1]{fontenc}
%\setmonofont{DejaVu Sans Mono}

%***********************************************************************************************
% Custom Color
%***********************************************************************************************
\definecolor{gray}{rgb}{0.9,0.9,0.9} %for source code display
\definecolor{dkgreen}{rgb}{0.1333,0.5451,0.1333}

%***********************************************************************************************
% Header and Footer
%***********************************************************************************************
\usepackage{fancyhdr} % Required for custom headers

\fancypagestyle{firstpage}{%
  
  \fancyhf{} % clear all header and footer fields
  \fancyfoot[L]{\small{\bf{\thepage}}}
  \fancyfoot[R]{\Author\ (\AuthorInitials)}
  \renewcommand{\headrulewidth}{0.0pt}%
}

%\pagestyle{fancy}
\fancypagestyle{normalpage}{%
    \fancyhf{} % clear all header and footer fields
    \fancyhead[C]{\small{\Subject}}
    \fancyfoot[L]{\small{\bf{\thepage}}}
    \fancyfoot[R]{\Author\ (\AuthorInitials)}
    
    \renewcommand{\headrulewidth}{0.4pt}
    \renewcommand{\footrulewidth}{0.4pt}
    
    \renewcommand{\headsep}{25pt}
}

%\AtBeginDocument{
%    \thispagestyle{firstpage}
%    }

%***********************************************************************************************
% Source code display Setting
%***********************************************************************************************
\usepackage{listings}
\usepackage{courier}

\lstloadlanguages{Python}
\lstset{
     language=Python,
     keywords={from,import,break,case,catch,continue,else,elseif,end,for,def,global,if,otherwise,persistent,return,switch,try,not,while,True,False,in},
     basicstyle=\footnotesize\ttfamily,
     tabsize=2,                   
     extendedchars=true,         
     breaklines=true,            
     keywordstyle=\color{blue},
     frame=none,                    % tb-for top and bottom, single-for around         
     stringstyle=\color{red}\ttfamily, 
     commentstyle=\color{dkgreen},
     backgroundcolor=\color{gray},
     showspaces=false,           
     showtabs=false,             
     xleftmargin=17pt,
     framexleftmargin=17pt,
     framexrightmargin=5pt,
     framexbottommargin=4pt,     
     showstringspaces=false 
}
% Creates a new command to include a C source, the first parameter is the filename of the script (without .c), the second parameter is the caption
\newcommand{\PythonSource}[2]{
\begin{itemize}
\item[]\lstinputlisting[caption=#2,label=#1]{#1.py}
\end{itemize}
}

%***********************************************************************************************
% Figure 
%***********************************************************************************************
%\MySubFigureTwo{./data/IDAC_hist.pdf}{./data/IDAC_hist_2.pdf}{Dynamic}{HuaXin}{0.2in}{IDAC Histogram}{fig1}{0.5}
\newcommand{\MySubFigureTwo}[8]
{
    \vspace{#5}
    \begin{figure}[H]
    \centering
    \subfigure[#3]
    {
        \includegraphics[scale=#8]{#1}
        \label{fig:a#7}
    }
    \subfigure[#4]
    {
        \includegraphics[scale=#8]{#2}
        \label{fig:b#7}
    }
    \caption[#6]{#6}
    \label{fig:#7}
    \end{figure}
}


%***********************************************************************************************
% Document Begin
%***********************************************************************************************
\begin{document}

\maketitle

%------------------------------------ %
%                                     %
%--------START YOUR MEMO HERE-------- %
%                                     %
%------------------------------------ %
\section{INTRODUCTION}
This memo documented the method to quickly deploy the python programs from our development machine to our target machine.
We suppose the target machine is Windows OS without internet network connection, 
like the PCs used in our manufacturing line or any PCs behind China firewall.
And the development machine can be any OS we prefer.
First, we need understand the several basic python libraries:
\begin{itemize}
    \item
    \textbf{setuptools} 
    \item
    \textbf{distribute} 
    \item
    \textbf{easy\_install} 
    \item
    \textbf{pip} 
    \item
    \textbf{virtualenv} 
\end{itemize}
The detail information for python module distribution can refer to:
\begin{lstlisting}
http://docs.python.org/2/distutils/index.html 
\end{lstlisting}
And we can download windows msi installation files for above packages from:
\begin{lstlisting}
http://www.lfd.uci.edu/~gohlke/pythonlibs/
\end{lstlisting}

\newpage
\section{BEST PRACTISE}
\subsection{Development Machine}
\begin{itemize}
\item
We need setup the virtual environment and test our code, so first, install the virtualenv.
\begin{lstlisting}
pip install virtualenv
virtualenv --help
\end{lstlisting}
\item
Before we start the virtualenv, we can use pip command to check the default system packages.
\begin{lstlisting}
pip list
\end{lstlisting}
All installed packages of system will be listed.
\item
Then we can create the source code dir, and start virtual environment.
\begin{lstlisting}
mkdir myproject
virtualenv --no-site-packages myproject 
cd myproject
source bin/activate
\end{lstlisting}
For Windows, I recommend to use PowerShell to start the virtualenv. First we need allow the PowerShell to execute scripts.
\begin{lstlisting}
PS C:\>get-executionpolicy                  # return the status of current script policy
Restricted
PS C:\>set-executionpolicy remotesigned     # set the script policy
PS C:\my_project\Scripts\>activate.ps1      # run the activate ps
\end{lstlisting}
\item
Now we are in the myproject virtual environment. We can use command deactivate to quit.
\begin{lstlisting}
deactivate
\end{lstlisting}
\item
We can use the pip command to check the dependencies again.
\begin{lstlisting}
pip list
\end{lstlisting}
\item
We can make a dir of source, and copy all source code files to "myproject/" under myproject.
And we can test the code, install the dependencies until the everything works OK under the virtual environment. \\
The dir structure looks like:
\begin{lstlisting}
--myproject
    |------bin/                     # dir of virtualenv, auto generated
    |------include/                 # dir of virtualenv, auto generated
    |------local/                   # dir of virtualenv, auto generated
    |------myproject/               # dir of own project 
            |--------__ini__.py
            |--------mysource.py
            |--------...
    |------requirments.txt          # pip generated file for dependencies
    |------setup.py                 # setup file for build egg
    |------README.md                # read me file
    |------dist/                    # dir of build egg, auto generated
    |------build/                   # dir of build egg, auto generated
\end{lstlisting}
\item
Now we can save the dependencies information into "requirements.txt". And we choose to bundle all the dependencies into a bundle file.
\begin{lstlisting}
pip freeze > requirements.txt
pip bundle myproject.pybundle -r requirements.txt
\end{lstlisting}
\item
Python called the project package "egg", like the ".deb" or ".rpm" packages to Linux software. 
To pack our own project "myproject" into an egg, we need create a setup.py file in the main dir.
The content in setup.py, for example:
\begin{lstlisting}
#!/usr/bin/env python
# encoding: utf-8

from setuptools import setup

setup(
    name="myproject",
    version="0.1",
    packages=["myproject", 
              "myproject/subproject"],
    package_data={'': ['*.dll', '*.so']},
    author='qibo',
    author_email='qibo@cypress.com',
    description='short description',
    long_description='long description',
    platforms='any',
    install_requires=[
        'numpy>=0.7',
        'matplotlib>=2.4',
    ]
    entry_points={
        "console_scripts": [
            'mpj = myproject.subproject:main'
        ]
)
\end{lstlisting}
The package\_data will include the files "*.dll" and "*.so". 
And we define a automation executable sript name "mpj", pointed to main function in myproject.subproject.
\item
For Eclipse user, PyDev will automatically create a "src" folder, and put the fold into the eviroment path. So we can create the dir structure like this:
\begin{lstlisting}
--myproject
    |------bin/                     # dir of virtualenv, auto generated
    |------include/                 # dir of virtualenv, auto generated
    |------local/                   # dir of virtualenv, auto generated
    |------src/
           |-----myproject/               # dir of own project 
                 |------__ini__.py
                 |------mysource.py
                 |------...
    |------requirments.txt          # pip generated file for dependencies
    |------setup.py                 # setup file for build egg
    |------README.md                # read me file
    |------dist/                    # dir of build egg, auto generated
    |------build/                   # dir of build egg, auto generated
\end{lstlisting}
\item
And add "src" path in the setup.py file, for example:
\begin{lstlisting}
#!/usr/bin/env python
# encoding: utf-8

from setuptools import setup
import sys

sys.path.append('src')

setup(
    name="myproject",
    version="0.1",
    packages_dir={'':'src'},
    packages=["myproject",
              "myproject/subproject"],
    package_data={'': ['*.dll', '*.so']},
    author='qibo',
    author_email='qibo@cypress.com',
    description='short description',
    long_description='long description',
    platforms='any',
    install_requires=[
        'numpy>=0.7',
        'matplotlib>=2.4',
    ]
    entry_points={
        "console_scripts": [
            'mpj = myproject.subproject:main'
        ]
)
\end{lstlisting}
\item
Now we can pack our own project into an .egg.
\begin{lstlisting}
python setup.py bdist_egg
\end{lstlisting}
\item
And also we can generate the source code package, or the windows .exe file.
\begin{lstlisting}
python setup.py bdist_egg       # egg file
python setup.py bdist_wininst   # windows .exe file
python setup.py sdist           # source code zip file
\end{lstlisting}
The released file will automatically generated in 'dist' fold.
\end{itemize}

\newpage
\subsection{Absolute Import}
It is strongly recommended to change the relative import to absolute import in source code.
For example, the source code structure of myproject:
\begin{lstlisting}
|-----myproject/               # dir of own project 
      |----packageA/
            |------__ini__.py
            |------moduleA.py
            |------moduleX.py
            |------...
      |----packageB/
            |------__ini__.py
            |------moduleB.py
            |------...
\end{lstlisting}
In moduleX.py, we want to import moduleA and moduleB, \\
for relative mode: 
\begin{lstlisting}
import .moduleA
import ..packageB.moduleB
\end{lstlisting}
for absolute mode (recommended):
\begin{lstlisting}
from myproject.packageA import moduleA
from myproject.packageB import moduleB
\end{lstlisting}
And for dll files, for absolute mode, we need get the absolute path first, for example, the driver of aardvark device, 
it contains 2 files, aardvark\_py.py and aardvark.dll, in aardvark\_py.py, we need load aardvark.dll. So we put the 2 files in one directory, and load the dll like this:
\begin{lstlisting}
import os
import sys
import inspect

# get current file's absolute path
dll_path = os.path.dirname(os.path.abspath(inspect.getfile(inspect.currentframe())))

# load the driver program from the absolute path
api = imp.load_dynamic('aardvark', dll_path + '/aardvark' + ext)
\end{lstlisting}
After changed to absolute mode, we need add the source path to PYTHONPATH under current virtualenv. For example, we have myproject v1.0 installed in our system, 
but we are debugging the myproject v1.1, we don't want the script will call the v1.0 instead of v1.1, so we need:
\begin{itemize}
    \item
    Start Virutalevn.
    For Windows:
    \begin{lstlisting}
    PS C:\my_project\Scripts\>activate.ps1      # run the activate ps
    \end{lstlisting}
    For Linux:
    \begin{lstlisting}
    source bin/activate
    \end{lstlisting}
    \item
    Add source path to PYTHONPATH.
    For Windows:
    \begin{lstlisting}
    $env:path="$env:Path;C:\Python27; D:\path\to\mycode;"
    \end{lstlisting}
    For Linux:
    \begin{lstlisting}
    export PYTHONPATH=$PYTHONPATH:~/path/to/mycode/
    \end{lstlisting}
\end{itemize}

\newpage
\subsection{Target Machine}
\subsubsection{Install Python Environment}
\begin{itemize}
\item
We need copy all the files to the target machine. We suppose this machine has no python installed.
\item
Then, install the original python. Here we choose "python-2.7.5.msi".
\item
Then, install the "setuptools-0.7.2.win32-py2.7.exe" and "pip-1.3.1.win32-py2.7.exe".
\item
After the installation, we need add installation paths to the environment variables, 
for example the default installation path, \verb/"C:\Python27\"/ and \verb/"C:\Python27\Scripts\"/.
\item
Reboot the machine.Done.
\end{itemize}
\subsubsection{Install Our Project}
\begin{itemize}
\item
Now it is very easy to install the dependencies of our project, open a console and type:
\begin{lstlisting}
pip install myproject.pybundle
\end{lstlisting}
\item
Install the myproject by double clicking the .exe file.
After installation, we can open python console, and check:
\begin{lstlisting}
import myproject
dir(myproject)
\end{lstlisting}
\item
Or we can directly run the code in a console by the entry point name,
in this example, we defined the name, "mpj", so we can type the name directly in console and run. \\
And done.
\end{itemize}


%------------------------------------ %
%                                     %
%--------END OF YOUR MEMO HERE------- %
%                                     %
%------------------------------------ %

\end{document}
%***********************************************************************************************
% Document End
%***********************************************************************************************
